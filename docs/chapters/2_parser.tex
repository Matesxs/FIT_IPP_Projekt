\section{Parser}


\subsection{Základní struktura}

Parsovací script se skládá z několika menších částí. \\
1) Deklarace regexů, lookup polí \\
2) Deklarace pomocných class (Instrukce, Argument) a funkcí \\
3) Entrypoint programu


\subsubsection{Pomocné funkce}

Úkolem pomocných funkcí je odstranit z hlavního toku programu repetitivní celky kódu a udržet ho relativně dobře čitelný.\\
Mezi takové patří hlavně funkce na kontrolu obsahu stringů pomocí regexů, sanitace vstupního stringu a jeho následné zpracování (např. odstranění nepotřebných znaků a převod speciálních znaků) a error handling.


\subsubsection{Pomocné classy}

Vlastní classy jsou využity na uskladnění a zpracování dat jednotlivých instrukcí před jejech výpisem na výstup. \\
Tato část programu není úplně nejdůležitější z pohledu funkčnosti samotného parseru, ale pomáhá s formátováním a generováním dané instrukce, což by bylo o dost náročnější bez těchto pomocných class.


\subsubsection{Entrypoint programu}

Entrypoint programu obsahuje hlavní flow programu, neboli kontrolu a zpracování argumentů, načítání vstupu a jeho předzpracování jako je odstranění komentářů, znaků nového řádku a jiných bílých znaků, parsování vstupních dat na jednotlivé komponenty instrukce, kontrolo vstupních dat a následné sestavéní instrukce do požadovaného formátu, tisknutí finální instrukce na výstup a samozřejmě také error handling.


\subsection{Argumenty programu}

Momentálně jediným podporovaným argumentem je \textbf{--help} nebo \textbf{-h} pro výpis krátkého popisu programu a jeho návratových kódu.


\subsection{Návratové hodnoty}

Pokud program projde bez chyby vrací návratovou hodnotu 0, jakékoli jiné hodnoty jsou kódy erroru.\\
1) 21 - Chybějící nebo špatný IPPcode22 header \\
2) 22 - Neznámá nebo nevalidní operace \\
3) 23 - Ostatní syntaktické errory


\subsection{Flow programu}

1) Kontrola a zpracování argumentů. \\
2) Načtení jednoho řádku ze standartního vstupu a jeho sanitace. \\
3) Header check pokud ještě nebyl nalezen. \\
4) Rozdělení načteného řádků na jednotlivé složky podle mezery nebo tabulátoru (instrukce a argumenty). \\
5) Kontrola argumentů podle zjištěné instrukce. \\
6) Tisk instrukce na výstup. \\
